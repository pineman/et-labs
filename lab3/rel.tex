\documentclass[a4paper, titlepage, portuguese]{article}
\usepackage[margin=2.5cm]{geometry}
%\usepackage{fontspec} % XeLaTeX
\usepackage[T1]{fontenc} % LaTeX
\usepackage[utf8]{inputenc} % LaTeX
%\usepackage{newtxmath, newtxtext}
\usepackage{csquotes}
\usepackage{babel}
%\usepackage[backend=bibtex]{biblatex}
%\usepackage[backend=biber]{biblatex}
%\addbibresource{bibliography.bib}

\usepackage{indentfirst}
\usepackage{graphicx}
	\graphicspath{{images/}}
\usepackage{grffile}
\usepackage{float}
\usepackage{amsmath}
	\allowdisplaybreaks
\usepackage{physics}
\usepackage{siunitx}
	\sisetup{inter-unit-product =\ensuremath{.}}
\usepackage{hyperref}

% Section styles
\renewcommand{\thesection}{}
\renewcommand{\thesubsection}{}
\renewcommand{\thesubsubsection}{}

% Useful commands
\newcommand{\eq}{\Leftrightarrow} % Equivalente
% Ordem de grandeza, e.g., "2\og{5}" => "2e5"
\newcommand{\og}[1]{{\times \num{e#1}}}
% Para numerar apenas uma equação
\newcommand\numberthis{\addtocounter{equation}{1}\tag{\theequation}}

% Header and footer
\usepackage{fancyhdr}
\pagestyle{fancy}
\fancyhf{}
\lhead{Electrotecnia Teórica}
\rhead{3º Laboratório}
\lfoot{IST - MEEC}
\rfoot{Página \thepage}
\renewcommand{\headrulewidth}{1pt}
\renewcommand{\footrulewidth}{0.5pt}

% Document
\begin{document}
	\begin{titlepage}
		\center
		\textsc{\bfseries\LARGE Instituto Superior Técnico}\\[1cm] % Name of your university/college
		\includegraphics[height=1.5cm]{IST_Logo.pdf}\\[2.5cm]

		\textsc{\large Engenharia Eletrotécnica e de Computadores}\\[0.5cm] % Major heading such as course name
		\textsc{\Large Eletrotecnia Teórica}\\[0.5cm] % Minor heading such as course title
		\textsc{\large 2017/2018 2º Semestre}\\[2cm]

		\rule{\textwidth}{1.6pt}\vspace*{-\baselineskip}\vspace*{2pt} % Thick horizontal line
		\rule{\textwidth}{0.4pt}\\[\baselineskip] % Thin horizontal line
			\textsc{\Huge \bfseries 3º Trabalho Laboratorial}\\[0.2cm]
			\bigskip
			\textsc{\large \bfseries circuito RLC série em regime forçado alternado sinusoidal}\\[0.2cm]
		\rule{\textwidth}{0.4pt}\vspace*{-\baselineskip}\vspace{3.2pt} % Thin horizontal line
		\rule{\textwidth}{1.6pt}\\[7cm]

		\begin{minipage}{0.9\textwidth}
			\begin{flushleft} \large
				\begin{Large}\bfseries\textsc{Autores:}\end{Large}\\[0.4cm]
				\begin{tabular}{l l l}
					Ricardo Simões			& 70389 & \normalsize ricardo.f.d.simoes@ist.utl.pt \\
					Rita Ramos					& 81616 & \normalsize rita.ramos@tecnico.ulisboa.pt \\
					João Pinheiro				& 84086 & \normalsize joao.castro.pinheiro@tecnico.ulisboa.pt \\
					João Sebastião			& 84087 & \normalsize joaofpsebastiao@tecnico.ulisboa.pt \\
				\end{tabular}
			\end{flushleft}
		\end{minipage}\\[0.5cm]

		\large \bfseries Laboratório segunda-feira, 09h30-11h30, Grupo D\\
		\large 16 de abril de 2018\\[1cm]
		\setcounter{page}{0}
	\end{titlepage}
	%\tableofcontents

	\section{3. Dimensionamento}
	\subsection{3.1}
    $$ \bar{Z} = \bar{Z}_{R_S} + (\bar{Z}_{R_L} + \bar{Z}_{L}) \parallel \bar{Z}_{C_d} $$
    \subsection{3.2}
    \subsubsection{a)}
    \subsubsection{b)}
    	\par
      Obtido no Matlab.
    \subsubsection{c)}
    \subsubsection{d)}

	%\begin{figure}[h]
	%	\centering
	%	\includegraphics[width=0.6\linewidth]{figure.pdf}
	%	\caption{An example figure}
	%	\label{fig:figure-example}
	%\end{figure}

	%Ver \autoref{fig:figure-example}

	%\printbibliography
\end{document}

