\documentclass[a4paper, titlepage, portuguese]{article}
\usepackage[margin=2.5cm]{geometry}
%\usepackage{fontspec} % XeLaTeX
\usepackage[T1]{fontenc} % LaTeX
\usepackage[utf8]{inputenc} % LaTeX
%\usepackage{newtxmath, newtxtext}
\usepackage{csquotes}
\usepackage{babel}
%\usepackage[backend=bibtex]{biblatex}
%\usepackage[backend=biber]{biblatex}
%\addbibresource{bibliography.bib}

\usepackage{indentfirst}
\usepackage{graphicx}
	\graphicspath{{images/}}
\usepackage{grffile}
\usepackage{float}
\usepackage{amsmath}
	\allowdisplaybreaks
\usepackage{physics}
\usepackage{siunitx}
	\sisetup{inter-unit-product =\ensuremath{.}, output-complex-root=\ensuremath{j},
		complex-root-position=before-number}
\usepackage{hyperref}

% Section styles
\renewcommand{\thesection}{}
\renewcommand{\thesubsection}{}
\renewcommand{\thesubsubsection}{}

% Useful commands
\newcommand{\eq}{\Leftrightarrow} % Equivalente
% Para numerar apenas uma equação
\newcommand\numberthis{\addtocounter{equation}{1}\tag{\theequation}}

% Header and footer
\usepackage{fancyhdr}
\pagestyle{fancy}
\fancyhf{}
\lhead{Electrotecnia Teórica}
\rhead{3º Laboratório}
\lfoot{IST - MEEC}
\rfoot{Página \thepage}
\renewcommand{\headrulewidth}{1pt}
\renewcommand{\footrulewidth}{0.5pt}

% Document
\begin{document}
	\begin{titlepage}
		\center
		\textsc{\bfseries\LARGE Instituto Superior Técnico}\\[1cm] % Name of your university/college
		\includegraphics[height=1.5cm]{IST_Logo.pdf}\\[2.5cm]

		\textsc{\large Engenharia Eletrotécnica e de Computadores}\\[0.5cm] % Major heading such as course name
		\textsc{\Large Eletrotecnia Teórica}\\[0.5cm] % Minor heading such as course title
		\textsc{\large 2017/2018 2º Semestre}\\[2cm]

		\rule{\textwidth}{1.6pt}\vspace*{-\baselineskip}\vspace*{2pt} % Thick horizontal line
		\rule{\textwidth}{0.4pt}\\[\baselineskip] % Thin horizontal line
			\textsc{\Huge \bfseries 3º Trabalho Laboratorial}\\[0.2cm]
			\bigskip
			\textsc{\large \bfseries circuito RLC série em regime forçado alternado sinusoidal}\\[0.2cm]
		\rule{\textwidth}{0.4pt}\vspace*{-\baselineskip}\vspace{3.2pt} % Thin horizontal line
		\rule{\textwidth}{1.6pt}\\[7cm]

		\begin{minipage}{0.9\textwidth}
			\begin{flushleft} \large
				\begin{Large}\bfseries\textsc{Autores:}\end{Large}\\[0.4cm]
				\begin{tabular}{l l l}
					Ricardo Simões			& 70389 & \normalsize ricardo.f.d.simoes@ist.utl.pt \\
					Rita Ramos					& 81616 & \normalsize rita.ramos@tecnico.ulisboa.pt \\
					João Pinheiro				& 84086 & \normalsize joao.castro.pinheiro@tecnico.ulisboa.pt \\
					João Sebastião			& 84087 & \normalsize joaofpsebastiao@tecnico.ulisboa.pt \\
				\end{tabular}
			\end{flushleft}
		\end{minipage}\\[0.5cm]

		\large \bfseries Laboratório segunda-feira, 09h30-11h30, Grupo D\\
		\large 16 de abril de 2018\\[1cm]
		\setcounter{page}{0}
	\end{titlepage}
	%\tableofcontents \newpage

	\section{3. Dimensionamento}
	\subsection{3.1}
	
		
	
		\begin{equation}
			\bar{Z} = \bar{Z}_{R_S} + \bar{Z}_{eq} + \bar{Z}_C,
		\end{equation}
		
		onde
		
		\begin{gather*}
			%my part
			\bar{Z}_{eq} = (\bar{Z}_{R_L} + \bar{Z}_{L}) \parallel \bar{Z}_{C_d} = \dfrac{\left(R_L + \mathrm{j} \omega L\right)\left(\dfrac{1}{\mathrm{j} \omega C_d}\right)}{(R_L + \mathrm{j} \omega L) + \dfrac{1}{\mathrm{j} \omega C_d}} \\
			%my part
			\bar{Z}_{eq} = \dfrac{1}{\dfrac{1}{R_L + \mathrm{j} \omega L} + \mathrm{j} \omega C_d} =
			\dfrac{1}{\dfrac{1 + (R_L + \mathrm{j} \omega L)(\mathrm{j} \omega C_d)}{R_L + \mathrm{j} \omega L}} =
			\dfrac{R_L + \mathrm{j} \omega L}{1 + (R_L + \mathrm{j} \omega L)(\mathrm{j} \omega C_d)} \\
			\implies \bar{Z} = R_S + \dfrac{R_L + \mathrm{j} \omega L}{1 + (R_L + \mathrm{j} \omega L)(\mathrm{j} \omega C_d)} + \dfrac{1}{\mathrm{j} \omega C} = \\
			= R_S + \dfrac{1}{\mathrm{j} \omega C} + \dfrac{\left(R_L + \mathrm{j} \omega L\right)\left(\dfrac{1}{\mathrm{j} \omega C_d}\right)}{R_L + \mathrm{j} \omega L + \dfrac{1}{\mathrm{j} \omega C_d}} = \\
			= R_S + \dfrac{1}{\mathrm{j} \omega C} + \dfrac{\left(\dfrac{R_L}{\omega L} + \mathrm{j}\right)\left(\dfrac{1}{\mathrm{j} \omega C_d}\right)}{\dfrac{R_L}{\omega L} + \mathrm{j} + \dfrac{1}{\mathrm{j} \omega^2 C_d L}} = \\
			= R_S + \dfrac{1}{\mathrm{j} \omega C} + \dfrac{\dfrac{1}{\omega C_d}}{\mathrm{j} + \dfrac{1}{\mathrm{j} \omega^2 C_d L}} = \\
			= R_S + \dfrac{1}{\mathrm{j} \omega C} + -\mathrm{j}\dfrac{1}{\omega C_d - \omega L} \\
			= R_S + -\mathrm{j}\underbrace{\left(\dfrac{1}{\omega C} + \dfrac{1}{\omega C_d - \omega L}\right)}_{= 0}
		\end{gather*}
		
	\subsection{3.2}
		\par
		A equação do circuito no domínio do tempo são:
		\begin{gather*}
			u_G = u_R + u_L + u_C = Ri + L\dv{i}{t} + \dfrac{1}{C} \int{i\dd{t}}
		\end{gather*}

		\par
		E em amplitudes complexas:
		\begin{gather*}
			\bar{U}_G = \bar{U}_R + \bar{U}_L + \bar{U}_C = R\bar{I} + \mathrm{j} \omega L - \dfrac{\mathrm{j}}{\omega C}\bar{I} \eq \\
			\eq \dfrac{\bar{U}_G}{\bar{I}} = \bar{Z} = R + \mathrm{j}(\omega L - \dfrac{1}{\omega C})
		\end{gather*}

	\subsubsection{a)}
		\par
		Na ressonância, $u_G$ e $i$ estão em fase, ou seja a impedância é real, logo:
		\begin{gather*}
			\omega_0 L = \dfrac{1}{\omega_0 C} \eq \\
			\eq C = \dfrac{1}{\omega_0^2 L} = \dfrac{1}{(2\pi \times \num{40e3})^2 \times \num{2e-3}} \approx \SI{7.916}{\nano\farad}
		\end{gather*}

	\subsubsection{b)}
		\par
		Obtido no Matlab.

	\subsubsection{c)}

	\subsubsection{d)}

	%\begin{figure}[h]
	%	\centering
	%	\includegraphics[width=0.6\linewidth]{figure.pdf}
	%	\caption{An example figure}
	%	\label{fig:figure-example}
	%\end{figure}

	%Ver \autoref{fig:figure-example}

	%\printbibliography
\end{document}

