% !TEX encoding = UTF-8 Unicode
\documentclass[portuguese, a4paper, titlepage]{article}
\usepackage[margin=2.5cm]{geometry}
%\usepackage{fontspec} % XeLaTeX
\usepackage[T1]{fontenc} % LaTeX
\usepackage[utf8]{inputenc} % LaTeX
%\usepackage{newtxmath, newtxtext}
%\usepackage{lmodern}
\usepackage{csquotes}
\usepackage{babel}
%\usepackage[backend=bibtex]{biblatex}
%\usepackage[backend=biber]{biblatex}
%\addbibresource{bibliography.bib}

\usepackage{indentfirst}
\usepackage{graphicx}
	\graphicspath{{images/}}
\usepackage{grffile}
\usepackage{float}
\usepackage{amsmath}
	\allowdisplaybreaks
\usepackage{commath}
\usepackage{amssymb}
\usepackage{mathtools}
\usepackage{siunitx}
	\sisetup{inter-unit-product =\ensuremath{.}}
\usepackage{hyperref}

% Section styles
%\renewcommand{\thesection}{\Roman{section}}
%\renewcommand{\thesubsection}{\alph{subsection})}
%\renewcommand{\thesubsubsection}{\roman{subsubsection}.}
\renewcommand{\thesection}{}
\renewcommand{\thesubsection}{}
\renewcommand{\thesubsubsection}{}

% Useful commands
\newcommand{\eq}{\Leftrightarrow} % Equivalente
% Ordem de grandeza, e.g., "2\og{5}" => "2e5"
\newcommand{\og}[1]{{\times \num{e#1}}}
% Em um ponto, e.g. "f(x)\at{x=5}" = f(x)|x=5
\newcommand{\at}[1]{\left.\right|_{#1}}
 % Para numerar apenas uma equação
\newcommand\numberthis{\addtocounter{equation}{1}\tag{\theequation}}
\DeclareMathOperator{\Div}{div}
\DeclareMathOperator{\Rot}{rot}
\newcommand{\real}{\ensuremath{\mathds{R}}}
 % Operator "d", e.g., "\frac{\dx{f}}{\dx{x}} = "df/dx"
\newcommand{\dx}[1]{\ensuremath{\operatorname{d}\!{#1}}}

% Header and footer
\usepackage{fancyhdr}
\pagestyle{fancy}
\fancyhf{}
\lhead{Electrotecnia Teórica}
\rhead{1º Laboratório}
\lfoot{\small Engenharia Eletrotécnica e de Computadores - IST}
\rfoot{Página \thepage}
\renewcommand{\footrulewidth}{0.5pt}

% Document
\begin{document}
	\hypersetup{pageanchor=false}
	\begin{titlepage}
		\center
		\textsc{\bfseries\LARGE Instituto Superior Técnico}\\[1cm] % Name of your university/college
		\includegraphics[height=1.5cm]{IST_Logo.pdf}\\[2.5cm]

		\textsc{\large Engenharia Eletrotécnica e de Computadores}\\[0.5cm] % Major heading such as course name
		\textsc{\Large Eletrotecnia Teórica}\\[0.5cm] % Minor heading such as course title
		\textsc{\large 2017/2018 - 2º Ano - 2º Semestre}\\[2cm]

		\rule{\textwidth}{1.6pt}\vspace*{-\baselineskip}\vspace*{2pt} % Thick horizontal line
		\rule{\textwidth}{0.4pt}\\[\baselineskip] % Thin horizontal line
			\textsc{\Huge \bfseries 1º Trabalho Laboratorial}\\[0.2cm]
			\bigskip
			\textsc{\large \bfseries Determinação Experimental Da Matriz De Coeficientes De Capacidade De Um Sistema De N+1 Condutores \\ (Via Analogia Reo-eléctrica)}\\[0.2cm]
		\rule{\textwidth}{0.4pt}\vspace*{-\baselineskip}\vspace{3.2pt} % Thin horizontal line
		\rule{\textwidth}{1.6pt}\\[5cm]

		\begin{minipage}{0.9\textwidth}
			\begin{flushleft} \large
				\begin{Large}\bfseries\textsc{Autores:}\end{Large}\\[0.4cm]
				\begin{tabular}{l l l}
					Ricardo Simões			& 70389 & \normalsize ricardo.f.d.simoes@ist.utl.pt \\
					Rita Ramos					& 81616 & \normalsize rita.ramos@tecnico.ulisboa.pt \\
					João Pinheiro				& 84086 & \normalsize joao.castro.pinheiro@tecnico.ulisboa.pt \\
					João Sebastião			& 84087 & \normalsize joaofpsebastiao@tecnico.ulisboa.pt \\
				\end{tabular}
			\end{flushleft}
		\end{minipage}\\[0.5cm]

		\large \bfseries Laboratório segunda-feira, 09h30-11h30\\
		\large 12 de março de 2018\\[1cm]
	\end{titlepage}
	\hypersetup{pageanchor=true}

	\section{Dimensionamento}
	\subsection{a)}

	\subsection{b)}
	Segundo a analogia reo-elétrica, é possível estabelecer uma relação entre a capacidade de um condensador e a condutância elétrica quando a geometria do sistema é semelhante.
	Sabe-se que na presença de uma corrente estacionária, a intensidade de corrente é dada por:

	\begin{align*}
		\int _ { S } \vec{J} \cdot \vec{n} d S = I
	\end{align*}

	Na situação do ensaio da figura 2, a intensidade do campo elétrico é constante e este é paralelo à normal, obtendo assim:

	\begin{align*}
		\vec{J} = \sigma \vec{E} \Rightarrow \int _ { S } \vec{J} \cdot \vec{n} d S = \int _ { S } \sigma \vec{E} \cdot \vec{n} d S  = \sigma E S = I
	\end{align*}

	Na presença de um campo elétrico estático, pela aplicação da Lei de Gauss, tem-se:

	\begin{align*}
			\int _ { S } \vec{D} \cdot \vec{n} d S = Q
	\end{align*}

	Do mesmo modo, devido ao facto de a intensidade do campo elétrico ser constante e paralelo à normal, tem-se:

	\begin{align*}
		\vec{J} = \epsilon \vec{E} \Rightarrow \int _ { S } \vec{D} \cdot \vec{n} d S = \int _ { S } \epsilon \vec{E} \cdot \vec{n} d S = \epsilon E S = Q
	\end{align*}

	Relacionando as expressões da capacidade de um condensador e condutância elétrica com as expressões obtidas acima, obtém-se:

	\begin{align*}
		&C =  \frac{Q}{U} = \frac{\epsilon E S}{U} \\
		&G = \frac{I}{U} = \frac{\sigma E S}{U} \\ 
		\Rightarrow &\frac{G}{C} = \frac{I}{Q} = \frac{\sigma}{\epsilon} \Rightarrow Q = I\frac{\epsilon}{\sigma}
	\end{align*}
	
	Recorrendo ao método das imagens e usando a expressão obtida no laboratório inicial para o cálculo dos potenciais, tem-se:
	
	\begin{align*}
		&V =  \frac{q}{2\pi\epsilon} \ln\left(\frac{r'}{r}\right) \\
		\eq &V = \frac{Q}{2\pi\epsilon l} \ln\left(\frac{r'}{r}\right)
	\end{align*}
	
	Por aplicação da Lei de Ohm e fazendo as respetivas substituições, obtém-se finalmente:
	
	\begin{align*}
		&R = \frac{V}{I} = \frac{\frac{Q}{2\pi\epsilon l} \ln\left(\frac{r'}{r}\right)}{I} = \frac{\frac{I\frac{\epsilon}{\sigma}}{2\pi\epsilon l} \ln\left(\frac{r'}{r}\right)}{I} = \frac{1}{2\pi\sigma l}\ln\left(\frac{r'}{r}\right)
	\end{align*}
	
	\subsection{c)}

	\subsection{d)}

	\subsection{e)}

	%\begin{figure}[h]
	%	\centering
	%	\includegraphics[width=0.6\linewidth]{figure.pdf}
	%	\caption{An example figure}
	%	\label{fig:figure-example}
	%\end{figure}

	%Ver \autoref{fig:figure-example}
\end{document}
